\section{Particle-mesh method}
The particle-mesh method can be described as the following sequence of four steps:
\begin{enumerate}
    \item Assign masses to mesh points,
    \item Solve the field equation (\autoref{eq:poisson}) on the mesh,
    \item Calculate the field strength at mesh-points,
    \item Find forces applied to individual particles by interpolation.
\end{enumerate}
In this section, each of these steps will be described in more detail.

\subsection{Mass assignment}\label{subsec:mass-assignment}
The specifics of assigning mass from particles to mesh points depend on the density profile (or \textit{shape}) associated with the particles.
In general, the particles need not be represented as idealized dimensionless points;
indeed, it is possible to construct a hierarchy of shapes, where each successive member covers a larger number of mesh points and whose application leads to smaller numerical errors.

An infinite hierarchy of shapes with this property, as described by Hockney and Eastwood in \cite{Hockney1988}, can be generated by successive convolutions with the "top-hat" function $\Pi$, defined as
\begin{equation*}
    \Pi(x) = \begin{cases}
        1,           & |x| < \frac{1}{2} \\
        \frac{1}{2}, & |x| = 1           \\
        0,           & \text{otherwise}.
    \end{cases}
\end{equation*}
The three most popular assignment schemes that hail from this family (and the ones implemented in our program) are the \textit{nearest grid point} (NGP), \textit{cloud in cell} (CIC), and \textit{triangular shaped cloud} (TSC) schemes, with shapes $S$ given by
\begin{align*}
    S_\text{NGP} & = \delta(x), & S_\text{CIC} & = \delta(x) * \frac{1}{H} \Pi\left(\frac{x}{H}\right) = \frac{1}{H}\Pi\left(\frac{x}{H}\right), & S_\text{TSC} & = \frac{1}{H}\Pi\left(\frac{x}{H}\right) * \frac{1}{H}\Pi\left(\frac{x}{H}\right) = \frac{1}{H}\Lambda \left(\frac{x}{H}\right),
\end{align*}
where $\Lambda$ is the triangle function
\begin{equation*}
    \Lambda(x) = \begin{cases}
        1 - |x|, & |x| < 1           \\
        0,       & \text{otherwise}.
    \end{cases}
\end{equation*}

In the one-dimensional case, the fraction of mass $W_p$ assigned to mesh-point $p$ from particle at position $x$ is given by
\begin{equation*}
    W(x-x_p) = W_p(x) = \int_{x_p-H/2}^{x_p+H/2} S(x'-x)dx'.
\end{equation*}
A simple rule for relating the assignment function $W$ defined above with shape $S$ can be found by noticing that
\begin{equation*}
    W(x) = \int_{-H/2}^{H/2}S(x'-x)dx' = \int_{-\infty}^\infty \Pi\left(\frac{x'}{H}\right)S(x'-x)dx' = \Pi\left(\frac{x}{H}\right) * S(x).
\end{equation*}
This implies that
\begin{align}\label{eq:assignment-functions}
    W_\text{NGP}(x) & = \Pi\left(\frac{x}{H}\right), & W_\text{CIC}(x) & = \Lambda\left(\frac{x}{H}\right), & W_\text{TSC}(x) & = \Pi\left(\frac{x}{H}\right) * \frac{1}{H} \Lambda\left(\frac{x}{H}\right) = (\Pi * \Lambda)\left(\frac{x}{H}\right).
\end{align}
Splitting the domain of integration in the expression for $W_\text{TSC}$ into five disjoint intervals shows that
\begin{equation*}
    (\Pi * \Lambda)(x) = \begin{cases}
        \frac{1}{8}(3-2|x|)^2, & \frac{1}{2} \leq |x| < \frac{3}{2} \\
        \frac{3}{4}-x^2,       & |x| < \frac{1}{2}                  \\
        0,                     & \text{otherwise}.
    \end{cases}
\end{equation*}

Two- and three-dimensional versions of the assignment functions in \autoref{eq:assignment-functions} are products of the assignment functions in each dimension.
For example, the three-dimensional assignment function $W$ is
\begin{equation*}
    W(\mathbf{x}) = W(x)W(y)W(z).
\end{equation*}
Hence, the mass assigned at mesh-point at $\mathbf{x}_\mathbf{p}$ is
\begin{equation*}
    m(\mathbf{x}_\mathbf{p}) = \sum_i m_i W_\mathbf{p}(\mathbf{x}_i),
\end{equation*}
or, in terms of density $\rho$,
\begin{equation}\label{eq:density-assignment}
    \rho(\mathbf{x}_\mathbf{p}) = \frac{1}{V} \sum_i m_i W_\mathbf{p}(\mathbf{x}_i),
\end{equation}
where $V = H^3$ is the volume of a cell and $i$ indexes the particles.

Obviously, \autoref{eq:density-assignment} is not suitable for direct application in the actual algorithm.
Instead, we iterate over all particles, identify the parent cell $\mathbf{p}$ of each particle (and its neighborhood) and update $\rho$.
This process is illustrated in \autoref{alg:density-assignment}.
\begin{algorithm}
    \caption{Density assignment algorithm}\label{alg:density-assignment}
    \begin{algorithmic}[1]
        \ForAll {particle $i$}
        \ForAll {cell $\mathbf{q}$ in $\mathcal{C}_S(\mathbf{x}_i)$}
        \State $\rho(\mathbf{x}_\mathbf{q}) \gets \rho(\mathbf{x}_\mathbf{q}) + m_i W(\mathbf{x}_i - \mathbf{x}_\mathbf{q}) / V$
        \EndFor
        \EndFor
    \end{algorithmic}
\end{algorithm}
The set $\mathcal{C}_S(\mathbf{x}_i)$ of cells that have to be considered while assigning density from the $i$-th particle, depends on the shape $S$ of the particle.
Specifically, we have $\mathcal{C}_\mathrm{NGP}(\mathbf{x}) = \{[\mathbf{x} / H]\}$, $\mathcal{C}_\mathrm{CIC}(\mathbf{x}) = \{\lfloor \mathbf{x}/H \rfloor + \mathbf{t} \;|\; t_i =0,1\}$, and $\mathcal{C}_\mathrm{TSC}(\mathbf{x}) = \{[\mathbf{x} / H] + \mathbf{t} \;|\; t_i = -1, 0, 1\}$.
It follows that $|\mathcal{C}_\mathrm{NGP}(\mathbf{x})| = 1$, $|\mathcal{C}_\mathrm{CIC}(\mathbf{x})| = 8$, and $|\mathcal{C}_\mathrm{TSC}(\mathbf{x})| = 27$ which illustrates the increasing computational cost resulting from using higher-order assignment schemes.
We note that \autoref{alg:density-assignment} can be parallelized if atomic increments are used in line 3.

\subsection{Solving the field equation}
The Poisson equation (\autoref{eq:poisson}) can be restated in integral form
\begin{equation*}
    \phi(\mathbf{x}) = \int G(\mathbf{x}-\mathbf{x}')\rho(\mathbf{x}') dV',
\end{equation*}
which has the following discrete analogue
\begin{equation}\label{eq:poisson-discrete}
    \phi(\mathbf{x}_\mathbf{p}) = V \sum_{\mathbf{p}'} G(\mathbf{x}_\mathbf{p} - \mathbf{x}_{\mathbf{p}'}) \rho(\mathbf{x}_{\mathbf{p}'}),
\end{equation}
where $G$ is the Green's function (potential due to unit mass).
The right-hand side of \autoref{eq:poisson-discrete} is a convolution sum that runs over a finite set of mesh points.
If we assume periodic boundary conditions, we can apply the discrete Fourier transform to both sides and use the convolution theorem to conclude that\footnote{
    In this work, the Hockney \& Eastwood definition of DFT is used, i.e.
    \begin{equation*}
        D(x_p) = \frac{1}{L}\sum_{l=0}^{N-1}\hat{D}(k)e^{ikx_p}, \quad \hat{D}(k) = H\sum_{p=0}^{N-1}D(x_p)e^{-ikx_p},
    \end{equation*}
    where $x_p = pH$.
    The conversion between this form and another popular definition,
    \begin{equation}\label{eq:standard-dft}
        \widetilde{D_H}(k) = \sum_{p=0}^{N-1}D_H(p)e^{-i2\pi kp / N},
    \end{equation}
    is given by
    \begin{equation*}
        \widetilde{D_H}(k) = \frac{1}{H}\hat{D}\left(\frac{2\pi}{NH}k\right),
    \end{equation*}
    where $D_H(p) = pH$.
}
\begin{equation}\label{eq:poisson-fourier-product}
    \hat{\phi}(\mathbf{k}) = \hat{G}(\mathbf{k}) \hat{\rho}(\mathbf{k}).
\end{equation}

An approximation to $\hat{G}$ can be found using a discretized version of the Laplacian in \autoref{eq:poisson-discrete}.
Specifically, for a 7-point stencil,
\begin{equation*}
    \begin{split}
        4\pi G\rho(\mathbf{x}_{ijk})
         & =\frac{\phi(\mathbf{x}_{i-1,j,k}) - 2\phi(\mathbf{x}_{ijk})+\phi(\mathbf{x}_{i+1,j,k})}{H^2}   \\
         & + \frac{\phi(\mathbf{x}_{i,j-1,k}) - 2\phi(\mathbf{x}_{ijk})+\phi(\mathbf{x}_{i,j+1,k})}{H^2}  \\
         & + \frac{\phi(\mathbf{x}_{i,j,k-1}) - 2\phi(\mathbf{x}_{ijk})+\phi(\mathbf{x}_{i,j,k+1})}{H^2}.
    \end{split}.
\end{equation*}
After Fourier-transforming both sides, applying the shift theorem, and simplifying using Euler's formula, we arrive at an expression for $\hat{\phi}$,
\begin{equation*}
    \hat{\phi}(\mathbf{k}) = \underbrace{-4\pi G\frac{(H/2)^2}{\sin^2\left(\frac{Hk_1}{2}\right)+\sin^2\left(\frac{Hk_2}{2}\right)+\sin^2\left(\frac{Hk_3}{2}\right)}}_{\hat{G}(\mathbf{k})} \hat{\rho}(\mathbf{k}),
\end{equation*}
where $\hat{G}$ can be identified by comparison with \autoref{eq:poisson-fourier-product}.
In the implementation, values of $\hat{G}$ should be computed only once and saved for future look-up.

\subsection{Field strength calculation}
The strength $\mathbf{g}$ of the gravitational field at mesh-point $\mathbf{x}_\mathbf{p}$ can be approximated using a central difference.
Our implementation currently supports two types of finite differences, described below.

The two-point finite difference operator $\mathbf{D}$, whose $x$ component is given by
\begin{equation}\label{eq:two-point-central-diff}
    D_x(\phi)(\mathbf{x_\mathbf{p}}) = - \frac{\phi(\mathbf{x}_{i+1,j,k}) - \phi(\mathbf{x}_{i-1,j,k})}{2H}
\end{equation}
(and analogously for the $y$ and $z$ components), is second order accurate.

The fourth-order accurate finite difference is given by
\begin{equation}\label{eq:four-point-central-diff}
    D_x(\phi)(\mathbf{x}_\mathbf{p}) = -\alpha\frac{\phi(\mathbf{x}_{i+1,j,k}) - \phi(\mathbf{x}_{i-1,j,k})}{2H} - (1-\alpha)\frac{\phi(\mathbf{x}_{i+2,j,k}) - \phi(\mathbf{x}_{i-2,j,k})}{4H},
\end{equation}
where $\alpha = 4/3$.


\subsection{Interpolation}
The value of the field strength $\mathbf{g}(\mathbf{x})$ at the position particle's position $\mathbf{x}$ is calculated by interpolating the values of $\mathbf{g}$ from the neighboring mesh-points.
Formally,
\begin{equation*}
    \mathbf{g}(\mathbf{x}) = \sum_\mathbf{p} W(\mathbf{x} - \mathbf{x}_\mathbf{p}) \mathbf{g}(\mathbf{x}_\mathbf{p}).
\end{equation*}
In practice, there is no need to sum over all mesh points.
Instead, we use an algorithm analogous to \autoref{alg:density-assignment} to only include the cells with non-zero contribution to the sum.
The method is illustrated in \autoref{alg:interpolation}.
\begin{algorithm}
    \caption{Field strength interpolation}\label{alg:interpolation}
    \begin{algorithmic}
        \ForAll {particle $i$}
        \ForAll {cell $\mathbf{q}$ in $\mathcal{C}_S(\mathbf{x}_i)$}
        \State $\mathbf{g}(\mathbf{x}_i) \gets \sum_\mathbf{q} W(\mathbf{x}_i - \mathbf{x}_\mathbf{q}) \mathbf{g}(\mathbf{x}_\mathbf{q})$
        \EndFor
        \EndFor
    \end{algorithmic}
\end{algorithm}
It is important to note that in order to retain correct physical behavior, the interpolation and mass assignment schemes must use the same shape to represent the particles.
The procedure in \autoref{alg:interpolation} is trivially parallelized by converting the sequential loop into a parallel one.

The procedures of density assignment and interpolation presented in \autoref{alg:density-assignment} and \autoref{alg:interpolation} are high level description.
More concrete formulations suitable for direct use in an implementation are given in \cite{Hockney1988} and \cite{Kravtsov2002PM}.