\section{Barnes-Hut algorithm}
The idea behind Barnes-Hut algorithm differs substantially from previously described mesh-based methods.
The algorithm deals with gravitational forces directly instead of deriving them from the mesh-defined potential, as was the case with the PM and \PThreeM{} methods.
Significant reduction of time complexity (from quadratic to $O(N \log N)$) is achieved by approximating ``far enough'' groups of particles by their center of mass (COM) \cite{trenti2008gravitationalnbodysimulations}.
The grouping of particles is hierarchical in nature and is thus best understood as a tree.
The entire set of particles comprises the top-level group, represented by the root of the tree;
the eight children of the root node are representative of groups of particles residing in each of the octants of the computational domain, etc.
The process of subdividing the space into eight smaller volumes at each node continues recursively until there is only one or zero particles left in a given volume.
Nodes which satisfy this condition are the leafs of tree and are sometimes called the \textit{external nodes}.
The remaining nodes, each of which has eight children, are called \textit{internal nodes}.

\subsection{Building the tree}
The data structure that fits the above description is called an \textit{octree}.
An internal node of the octree stores the COM vector and the total mass of the group it represents, whereas an external node stores a reference to the actual particles (or is empty if no particle was found in its associated volume).
The recursive procedure of building the tree is shown in \autoref{alg:bh-tree-insert}.
\begin{algorithm}
    \caption{Insert a particle into the Barnes-Hut tree}\label{alg:bh-tree-insert}
    \begin{algorithmic}
        \Function{Insert}{$n$, $p$}
        \If{$n$ is an internal node}
        \State Update $n.\textrm{COM}$ and total mass $n.M$ of $n$ with $p$
        \State \Call{Insert}{child of $n$ that should contain $p$, $p$}
        \ElsIf{$n$ is empty}
        \State Assign $p$ to $n$
        \Else \Comment{Occupied external node}
        \State Subdivide $n$ into child nodes
        \State Move existing particle $p'$ in $n$ into child that should contain $p'$
        \State Update center of mass and total mass of $n$ with $p$ and $p'$
        \State \Call{Insert}{child of $n$ that should contain $p$, $p$}
        \EndIf
        \EndFunction
    \end{algorithmic}
\end{algorithm}

\subsection{Force calculation}
The last method presented in this work is the Barnes-Hut algorithm.
In the Barnes-Hut algorithm, the net gravitational force on a particle $p$ is calculated by summing the contributions from single particles or groups of particles while traversing the tree.
The decision whether the force can be approximated using the information stored in an internal node $n$ depends on the relative distance from $p$ to $n.\textrm{COM}$ (the center of mass of group represented by $n$).
The distance is relative to the \textit{width} $H$ of the node, i.e. the side length of the cubical volume encompassed by the node.
More concretely, the approximation takes place if $n.H / |n.\mathrm{COM} - p.\mathbf{x}| < \theta$, where $\theta$ is the so-called \textit{opening angle}.
In the extreme case when $\theta$ is set to zero, no approximations take place, and the algorithm reduces to the PP method.
The procedure described above is illustrated in \autoref{alg:bh-find-force}.
\begin{algorithm}
    \caption{Compute gravitational force on a particle using Barnes-Hut approximation}
    \label{alg:bh-find-force}
    \begin{algorithmic}
        \Function{FindForce}{$n$, $p$, $\theta$}
        \If{$n$ is an external node}
        \If{$n$ contains a particle $q \neq p$}
        \State $p.\mathbf{F} \gets p.\mathbf{F} + \Call{Gravity}{q.\mathbf{x}, q.m, p.\mathbf{x}}$
        \EndIf
        \State \Return
        \EndIf
        \If{$n.H / |n.\text{COM} - p.\mathbf{x}| < \theta$}
        \State $p.\mathbf{F} \gets p.\mathbf{F} + \Call{Gravity}{n.\mathrm{COM}, n.M, p.\mathbf{x}}$
        \State \Return
        \EndIf
        \ForAll{child $n_c$ of $n$}
        \State \Call{FindForce}{$n_c$, $p$, $\theta$}
        \EndFor
        \EndFunction
    \end{algorithmic}
\end{algorithm}
In the implementation, the \textsc{Gravity} function calculates the gravitational force softened by $\epsilon$, i.e. it returns the value
\begin{equation*}
    \mathbf{F}^\textrm{soft}_{ij} = -G\frac{m_i m_j}{(r_{ij}^2 + \epsilon^2)^{3/2}}\mathbf{r}_{ij}.
\end{equation*}
The pairwise potential energy associated with this force is given by
\begin{equation}\label{eq:pe-soft}
    \Phi_{ij}^\textrm{soft} = - \frac{G m_i m_j}{\sqrt{r_{ij}^2 + \epsilon^2}}.
\end{equation}

We note that direct calculation of total potential energy is infeasible as $O(N^2)$ operations would be required.
Instead, we use an approximation based on the values stored in the tree.
The approximate value of the potential energy is accumulated for each particle using a procedure analogous to force calculation.
Indeed, the only difference between the two is the replacement of gravitational force calculation in \autoref{alg:bh-find-force} with potential energy calculation according to \autoref{eq:pe-soft}.

It is also noteworthy that the procedure outlined in \autoref{alg:bh-find-force} is embarrassingly parallel.
In our CPU implementation, the workload is split between an arbitrary number of threads on particle-by-particle basis.
On the other hand, the parallelization of the tree building procedure in \autoref{alg:bh-tree-insert} is far from trivial (see for example \cite{warren_salmon_1993}).
In this work, we will not be exploring this idea further.