\begin{abstract}
    \begin{center}
        Wybrane metody symulacji $N$-ciał
    \end{center}
    Niniejsza praca przedstawia implementację oraz analizę porównawczą trzech ważnych algorytmów używanych w symulacjach $N$-ciał;
    skoncentrowano się na metodach PM (ang. \textit{Particle-mesh}), \PThreeM{} (ang. \textit{Particle-particle particle-mesh}) oraz algorytmie Barnesa-Huta.
    Symulacje $N$-ciał odgrywają kluczową rolę w astrofizyce, gdzie używane są do modelowania systemów, w których dominującą siłą jest oddziaływanie grawitacyjne.
    Głównym celem pracy jest stworzenie autorskich wydajnych implementacji powyższych metod oraz zbadanie ich dokładności i wydajności obliczeniowej.
    Metoda PM została zaimplementowana zarówno w wersji na CPU, jak i GPU, przy czym wersja GPU osiągnęła znaczące przyspieszenie (nawet do 1200\%).
    Korekta krótkozasięgowa w metodzie \PThreeM{} została zrównoleglona na CPU, co pozwoliło uzyskać czterokrotne przyspieszenie względem wersji jednowątkowej.
    W pracy omówiono również implementację procedury konstrukcji drzewa ósemkowego, która zapewnia przyspieszenie budowy drzewa o 40\% oraz skrócenie o 15\% całkowitego czasu wykonania algorytmu Barnesa-Huta dzięki efektywnemu wykorzystaniu pamięci podręcznej.
    Poprzez serię testowych symulacji, obejmujących modelowanie galaktyki spiralnej, gromady kulistej oraz kolizji dwu galaktyk, przeprowadzono ocenę dokładności, kosztu obliczeniowego oraz skalowalności metod.
    Wyniki pracy dostarczają praktycznych informacji na temat zalet i ograniczeń poszczególnych metod.

    \vspace{1.5em}

    \noindent \textbf{Słowa kluczowe:} symulacje grawitacyjne N-ciał, metoda PM, metoda P3M, algorytm Barnesa-Huta, optymalizacja wydajności
\end{abstract}


