\chapter{Introduction}
The dominant force over large distances is the gravitational force.
The force exerted on a body with mass $m_2$ at the point $\mathbf{x}_2$ by a body with mass $m_1$ located at $\mathbf{x}_1$ can be expressed by the relation
\begin{equation}\label{eq:law-of-uni-grav}
    \mathbf{F} = -G\frac{m_1m_2}{|\mathbf{x}_{21}|^3}\mathbf{x}_{21}
\end{equation}
where $G$ is the gravitational constant $6.674\times 10^{-11} \mathrm{m}^3 \mathrm{kg}^{-1}\mathrm{s}^{-2}$ and $\mathbf{x}_{21} = \mathbf{x}_2 - \mathbf{x}_1$.
Therefore, the evolution of a system of $N$ bodies is described by $N$ equations
\begin{equation}\label{eq:pp-method}
    \ddot{\mathbf{x}}_i = -G\sum_{j\neq i} \frac{m_j}{|\mathbf{x}_{ij}|^3}\mathbf{x}_{ij}.
\end{equation}
for each $i = 1,\dots, N$.
Direct application of \autoref{eq:pp-method} is the basis of the so-called \textit{particle-particle} method.
The method is characterized by $O(N^2)$ time complexity (more precisely, it requires $(N-1)N/2$ operations if Newton's 3rd law is used in the computation).
Assuming that 100ns are required to perform the floating-point operations under the summation symbol, $N=30,000$, and 150 iterations, the simulation would take approximately 2 hours to complete.
Therefore, it is evident that more efficient algorithms are needed to make simulations of this scale feasible.

The \textit{particle-mesh} (PM) technique, introduced around 1985 by Hockney and Eastwood, was an early improvement over the PP method.
In the PM approach, the space is divided into a rectangular grid (or mesh) of cells.
Each cell is assigned a portion of the mass of nearby particles, creating a density distribution $\rho(\mathbf{x})$.
The relation between the density and gravitational potential $\phi$, in the form of Poisson's equation
\begin{equation}\label{eq:poisson}
    \nabla^2\phi = 4\pi G \rho,
\end{equation}
is then used to obtain the potential at each cell center.
The gravitational field $\mathbf{g}$ can then be calculated as $\mathbf{g} = -\nabla \phi$.
Since $\mathbf{g}$ equals the acceleration due to gravity, we get $\ddot{\mathbf{x}}_i = \mathbf{g}(\mathbf{x}_i)$.

The drawback of the PM method is its poor modeling of forces over short distances.
Eastwood and Hockney proposed a remedy for this problem: the \textit{particle-particle-particle-mesh} method (or \PThreeM{} in short).
In the \PThreeM{} method, the force on the $i$-th particle is split into two components: \textit{short-range} and \textit{long-range} force.
The long-range force is calculated using the PM method, whereas the short-range force can be found by direct summation of the forces due to nearby particles.

The computational complexity of the PM and \PThreeM{} methods depends on the implementation of the potential solver used to calculate $\phi$ from \autoref{eq:poisson}.
For instance, if a fast Fourier transform is used, then the complexity of the PM algorithm is $O(N + N_g^3\log N_g)$, where $N_g$ is the number of cells in a single dimension of the grid (note it is linear in $N$).
For the \PThreeM{} method, the worst-case scenario happens when all particles are clustered closely together, which causes the short-range $O(N^2)$ correction part to become dominant.
