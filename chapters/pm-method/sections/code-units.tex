\section{Code units}
Implementation of the PM (and \PThreeM{}) methods can be simplified by switching to a system of dimensionless units, often called \textit{code units}.
The natural units of time and length in a PM simulation are $H$ (cell width) and $\textrm{DT}$ (time step length), respectively.
Hence, length in a PM code is conveniently expressed in terms of multiples of $H$, and similarly, time intervals are given as a multiple of $\textrm{DT}$, i.e., the conversion relations are
\begin{equation*}
    x' = \frac{x}{H} \quad \text{and} \quad t' = \frac{t}{\textrm{DT}}.
\end{equation*}
From there, it follows that
\begin{equation*}
    v' = \frac{\textrm{DT}}{H}v \quad \text{and} \quad a' = \frac{\textrm{DT}^2}{H}a.
\end{equation*}
The expected relation $\mathbf{g}' = -\nabla' \phi'$ leads to the definition $\phi' = (\textrm{DT}^2 / H^2)\phi$.
By stipulating that we have $\nabla'^2\phi = \rho'$, we get $\rho' = \textrm{DT}^2 \cdot 4\pi G\rho$, $m' = (\textrm{DT}^2\cdot 4\pi G / H^3) m$, and $G' = 1/(4\pi)$.
