\section{Interpolation}
The value of the field strength $\mathbf{g}(\mathbf{x})$ at the position particle's position $\mathbf{x}$ is calculated by interpolating the values of $\mathbf{g}$ from the neighboring mesh-points.
Formally,
\begin{equation*}
    \mathbf{g}(\mathbf{x}) = \sum_\mathbf{p} W(\mathbf{x} - \mathbf{x}_\mathbf{p}) \mathbf{g}(\mathbf{x}_\mathbf{p}).
\end{equation*}
In practice, there is no need to sum over all mesh points.
Instead, we use an algorithm analogous to \autoref{alg:density-assignment} to only include the cells with non-zero contribution to the sum.
The method is illustrated in \autoref{alg:interpolation}.
\begin{algorithm}
    \caption{Field strength interpolation}\label{alg:interpolation}
    \begin{algorithmic}[1]
        \ForAll {particle $i$}
        \ForAll {cell $\mathbf{q}$ in $\mathcal{C}_S(\mathbf{x}_i)$}
        \State $\mathbf{g}(\mathbf{x}_i) \gets \sum_\mathbf{q} W(\mathbf{x}_i - \mathbf{x}_\mathbf{q}) \mathbf{g}(\mathbf{x}_\mathbf{q})$
        \EndFor
        \EndFor
    \end{algorithmic}
\end{algorithm}
It is important to note that in order to retain correct physical behavior, the interpolation and mass assignment schemes must use the same shape to represent the particles.
The procedure in \autoref{alg:interpolation} is trivially parallelized by converting the sequential loop into a parallel one.

The procedures of density assignment and interpolation presented in \autoref{alg:density-assignment} and \autoref{alg:interpolation} are high level description.
More concrete formulations suitable for direct use in an implementation are given in \cite{Hockney1988} and \cite{Kravtsov2002PM}.
