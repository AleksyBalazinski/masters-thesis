\section{Solving the field equation}\label{sec:solving-the-field-equation}
The Poisson equation (\autoref{eq:poisson}) can be restated in integral form
\begin{equation*}
    \phi(\mathbf{x}) = \int \mathcal{G}(\mathbf{x}-\mathbf{x}')\rho(\mathbf{x}') dV',
\end{equation*}
which has the following discrete analogue
\begin{equation}\label{eq:poisson-discrete}
    \phi(\mathbf{x}_\mathbf{p}) = V \sum_{\mathbf{p}'} \mathcal{G}(\mathbf{x}_\mathbf{p} - \mathbf{x}_{\mathbf{p}'}) \rho(\mathbf{x}_{\mathbf{p}'}),
\end{equation}
where $\mathcal{G}$ is the Green's function (potential due to unit mass).
The right-hand side of \autoref{eq:poisson-discrete} is a convolution sum that runs over a finite set of mesh points.
If we assume periodic boundary conditions, we can apply the discrete Fourier transform to both sides and use the convolution theorem to conclude that\footnote{
    In this work, the Hockney \& Eastwood definition of DFT is used, i.e.
    \begin{equation*}
        D(x_p) = \frac{1}{L}\sum_{l=0}^{N-1}\hat{D}(k)e^{ikx_p}, \quad \hat{D}(k) = H\sum_{p=0}^{N-1}D(x_p)e^{-ikx_p},
    \end{equation*}
    where $x_p = pH$.
    The conversion between this form and another popular definition,
    \begin{equation}\label{eq:standard-dft}
        \widetilde{D_H}(k) = \sum_{p=0}^{N-1}D_H(p)e^{-i2\pi kp / N},
    \end{equation}
    is given by
    \begin{equation*}
        \widetilde{D_H}(k) = \frac{1}{H}\hat{D}\left(\frac{2\pi}{NH}k\right),
    \end{equation*}
    where $D_H(p) = D(pH)$.
}
\begin{equation}\label{eq:poisson-fourier-product}
    \hat{\phi}(\mathbf{k}) = \hat{\mathcal{G}}(\mathbf{k}) \hat{\rho}(\mathbf{k}).
\end{equation}

An approximation to $\hat{\mathcal{G}}$ can be found using a discretized version of the Laplacian in \autoref{eq:poisson-discrete}.
Specifically, for a 7-point stencil,
\begin{equation*}
    \begin{split}
        4\pi G\rho(\mathbf{x}_{ijk})
         & =\frac{\phi(\mathbf{x}_{i-1,j,k}) - 2\phi(\mathbf{x}_{ijk})+\phi(\mathbf{x}_{i+1,j,k})}{H^2}   \\
         & + \frac{\phi(\mathbf{x}_{i,j-1,k}) - 2\phi(\mathbf{x}_{ijk})+\phi(\mathbf{x}_{i,j+1,k})}{H^2}  \\
         & + \frac{\phi(\mathbf{x}_{i,j,k-1}) - 2\phi(\mathbf{x}_{ijk})+\phi(\mathbf{x}_{i,j,k+1})}{H^2}.
    \end{split}.
\end{equation*}
Applying the discrete Fourier transform to both sides and using the shift theorem, we get
\begin{align*}
    4\pi G \hat{\rho}(\mathbf{k})
     & = \frac{1}{H^2}\sum_{i=1}^{3}\left( e^{-iHk_i} + e^{iHk_i}-2 \right)\hat{\phi}(\mathbf{k})       \\
     & = \frac{1}{H^2} \sum_{i=1}^{3}\left( e^{iHk_i/2} - e^{-iHk_i/2} \right)^2 \hat{\phi}(\mathbf{k}) \\
     & = -\frac{4}{H^2}\sum_{i=1}^{3}\sin^2\left(\frac{Hk_i}{2}\right)\hat{\phi}(\mathbf{k}).
\end{align*}
and hence
\begin{equation}\label{eq:dft-transformed-phi}
    \hat{\phi}(\mathbf{k}) = -4\pi G\underbrace{\frac{(H/2)^2}{\sin^2(Hk_1/2) + \sin^2(Hk_2/2) + \sin^2 (Hk_3/2)}}_{\hat{\mathcal{G}}(\mathbf{k})} \hat{\rho}(\mathbf{k}),
\end{equation}
where $\hat{\mathcal{G}}$ can be identified by comparison with \autoref{eq:poisson-fourier-product}.
It is worth noting that the constant multiplier $(-4\pi G)$ is often left out of $\hat{\mathcal{G}}$ (this is the convention used in \cite{Hockney1988}).
In the implementation, values of $\hat{\mathcal{G}}$ are computed only once and saved for future look-up.

In the one-dimensional case, the above discussion can be easily rephrased in terms of a diagonalization problem \cite{demanet2013fourier}.
In one dimension, the Poisson equation is
\begin{equation}\label{eq:poisson-1d}
    \frac{d^2 \phi}{dx^2} = \rho.
\end{equation}
The interval $[a, b]$ on which we wish to find $\phi$ is assumed to be discretized into $N$ points $x_j = a + jH$, where $H = (b - a) / (N - 1)$ and $j=0,\dots, N-1$.
Furthermore, we assume periodic boundary conditions so that $\phi(x_{-1}) = \phi(x_{N-1})$ and $\phi(x_{N}) = \phi(x_0)$.
The approximation of $\Delta$ at $x_j$ is
\begin{equation*}
    (\Delta_H \phi)_j \equiv \frac{\phi_{j-1} - 2\phi_j + \phi_{j+1}}{H^2}.
\end{equation*}
Thus, the discrete version of \autoref{eq:poisson-1d} reads $(\Delta_H \phi)_j = \rho_j$ or
\begin{equation*}
    \frac{\phi_{j-1}-2\phi_j + \phi_{j+1}}{H^2} = \rho_j,
\end{equation*}
where $\rho_j = \rho(x_j)$.
This gives a system of $N$ equations (one per each sampled value of $\rho$) with $N$ unknowns (values of $\phi$) of the following form
\begin{equation}\label{eq:poisson-1d-matrix}
    \underbrace{\frac{1}{H^2}
        \begin{bmatrix}
            2  & -1     &        &        & -1 \\
            -1 & 2      & \ddots &        &    \\
               & \ddots & \ddots & \ddots &    \\
               &        & \ddots & 2      & -1 \\
            -1 &        &        & -1     & 2
        \end{bmatrix}}_{\Delta_H}
    \underbrace{\begin{bmatrix}
            \phi_0     \\
            \phi_1     \\
            \vdots     \\
            \phi_{N-2} \\
            \phi_{N-1}
        \end{bmatrix}}_{\bar{\phi}}
    = -\underbrace{\begin{bmatrix}
            \rho_0     \\
            \rho_1     \\
            \vdots     \\
            \rho_{N-2} \\
            \rho_{N-1}
        \end{bmatrix}}_{\bar{\rho}}.
\end{equation}
Next we define $\bar{\psi}(k) \in \mathbb{C}^N$ by $\psi_j(k) \equiv e^{i2\pi jk/N} = \omega^{jk}$ for $j=0,\dots,N-1$.
The $j$-th row of $H^2\Delta_H \bar{\psi}$ equals
\begin{align*}
    (H^2\Delta_H \bar{\psi})_j
     & = -\psi_{j-1} + 2\psi_j - \psi_{j+1}
    = -\omega^{(j-1)k} + 2\omega^{jk} - \omega^{(j+1)k}     \\
     & = -\omega^{jk}(\omega^{-k} - 2 + \omega^k)
    = -\omega^{jk}(\omega^{k/2} - \omega^{-k/2})^2          \\
     & = 4\omega^{jk} \sin^2\left( \frac{\pi k}{N} \right).
\end{align*}
Hence we have $(H^2\Delta_H \bar{\psi})_j = 4\sin^2(\pi k/N) \psi_j$ which implies
\begin{equation*}
    \Delta_H \bar{\psi}(k)
    = \frac{4}{H^2} \sin^2\left( \frac{\pi k}{N} \right) \bar{\psi}(k).
\end{equation*}
This results tells us that for any value of $k$, $\bar{\psi}(k)$ is an eigenvector of $\Delta_H$ with eigenvalue $(4/H^2)\sin^2(\pi k/N)$.
For $k=0,\dots, N-1$ we get $N$ linearly independent eigenvectors $\{\bar{\psi}(k)\}$ which necessarily form the basis of $\mathbb{C}^N$.
This fact implies that $\Delta_H$ is diagonalizable.
The matrix $F$ of its eigenvectors is given by $F_{jk} = \omega^{jk}$, and the inverse is easily verified to be $F^{-1} = (1/N)F^\dagger$.
The eigendecomposition of $\Delta_H$ is therefore
\begin{equation*}
    \Delta_H = F\Lambda F^{-1},
\end{equation*}
where $\Lambda = \text{diag}((4/H^2)\sin^2(\pi k/N))$.
Substitution into \autoref{eq:poisson-1d-matrix} yields $F\Lambda F^{-1} \bar{\phi} = -\bar{\rho}$ or
\begin{equation*}
    F^{-1}\bar{\phi} = -\Lambda^{-1}F^{-1}\bar{\rho}.
\end{equation*}
Evaluating the left-hand side leads to the conclusion that $F^{-1}\bar\phi$ is nothing else but the DFT of $\phi$.
Indeed,
\begin{equation*}
    (F^{-1}\phi)_k
    = \frac{1}{N}(F^\dagger \phi)_k
    = \frac{1}{N}\sum_{j=0}^{N-1}F^\dagger_{kj}\phi_j
    = \frac{1}{N}\sum_{j=0}^{N-1}\omega^{-kj}\phi_j
    = \frac{1}{N}\sum_{j=0}^{N-1}e^{-2\pi i jk / N}\phi_j
\end{equation*}
which is exactly the discrete Fourier transform of $\phi$ (using the standard definition of the DFT).
Thus, we see that the DFT of the Green's function derived in \autoref{eq:dft-transformed-phi} (or rather the one-dimensional variant thereof) is the $k$-th eigenfunction of the discretized Laplacian (with slight differences due to a different definition of the DFT being used there).
By following the line of reasoning presented above, we observe a deep connection between the DFT and the eigendecomposition of the discretized Laplace operator.

A natural question that arises in the context of this discussion is whether a similar relation holds in the continuous case.
The answer turns out to be positive;
assuming free-space boundary conditions, the complex exponential $e^{2\pi i \mathbf{k} \cdot \mathbf{x}}$ (kernel of the inverse Fourier transform) is an eigenfunction of the Laplace operator (with eigenvalue $-(2\pi k)^2$) and the Fourier transform allows for a ``diagonalization'' of the Laplacian \cite{demanet2013fourier}.
More precisely, we have
\begin{equation*}
    \Delta = F \Lambda F^{-1},
\end{equation*}
where $F$ is the inverse FT, and $\Lambda = -(2\pi k)^2$.
Applying this result to the Poisson equation yields $F^{-1}\phi = \Lambda^{-1}F^{-1}\rho$ or
\begin{equation}\label{eq:poor-mans-poisson-solver}
    \hat\phi = -\frac{1}{4\pi^2 k^2} \hat\rho,
\end{equation}
where the circumflex denotes the standard Fourier transform and $k = |\mathbf{k}|$.
The function $-(2\pi k)^2$ is sometimes used instead of the Green's function $\mathcal{G}$ defined in \autoref{eq:dft-transformed-phi}.
This approach is the basis of the ``poor man's Poisson solver'' (as dubbed by \cite{Hockney1988}).
While implementing it, we have to take into account that in the standard definition of the DFT, nonnegative frequencies correspond to $0 \leq k \leq N/2$ and negative frequencies correspond to $N/2+1 \leq k \leq N-1$ (see \cite{press2007numerical}, pp. 607--608).
This necessitates the following ``index wrapping'':
\begin{equation*}
    k_i =
    \begin{cases}
        i     & \text{if } i \leq \frac{N}{2} \\
        i - N & \text{if } i > \frac{N}{2}
    \end{cases},
\end{equation*}
where $i$ is the index of the gridpoint.
